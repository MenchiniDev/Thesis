\chapter{Introduction}

\begin{abstract}

Questa tesi affronta il problema della navigazione autonoma di un rover in ambienti lunari 
mediante lo sviluppo di una pipeline integrata che combina modelli 
di Reinforcement Learning con metodi geometrici deterministici a basso costo computazionale. 
L’obiettivo principale è valutare se agenti RL, addestrati in \textit{Isaac Sim} 
tramite PPO, SAC e Recurrent PPO e ottimizzati mediante 
\textit{curriculum learning} possano fornire soluzioni più robuste, 
adattive e generalizzabili rispetto ai planner deterministici 
tradizionali, considerando sia la qualità dei percorsi generati 
sia la capacità di reagire a variazioni improvvise del terreno.

La pipeline percettiva è basata sulla stima della pendenza ottenuta 
da mappe di profondità e su un algoritmo di clustering geometrico 
interpretabile, progettato per sostituire o affiancare approcci classici 
di slope estimation. A tal fine viene proposto un confronto sistematico 
tra metodi deterministici rule--based e tecniche derivate da features 
apprese, valutando stabilità, continuità del segnale e impatto sul comportamento 
dei modelli RL.

Un aspetto centrale del lavoro riguarda i vincoli dell'\textit{edge computing}: 
sia le politiche RL sia la pipeline di percezione sono state sviluppate 
per funzionare su piattaforme a risorse limitate, 
come Raspberry Pi 5 e sistemi embedded analoghi. Questo ha richiesto 
scelte architetturali orientate alla minimizzazione del consumo energetico, 
alla riduzione della latenza, e all’impiego di pipeline numeriche leggere, 
prive di componenti neurali ad alto costo computazionale.

I modelli addestrati in simulazione sono stati valutati tramite metriche di stabilità del training, 
capacità di generalizzazione e robustezza dinamica, quindi trasferiti su piattaforma reale tramite ROS2, 
dove sono state analizzate prestazioni, 
latenza end--to--end e comportamento emergente in scenari fisici.

I risultati mostrano che l’integrazione tra RL, percezione geometrica ottimizzata 
per l’edge computing e simulazione realistica consente al rover di 
prendere decisioni più reattive e flessibili rispetto ai metodi deterministici, 
pur mantenendo un budget di calcolo ed energia compatibile con missioni robotiche in ambienti lunari. 
Il lavoro fornisce infine indicazioni su possibili estensioni future, 
incluse strategie di \textit{sim-to-real} più affidabili e nuove modalità di integrazione tra modelli RL e percezione tradizionale.


\end{abstract}
