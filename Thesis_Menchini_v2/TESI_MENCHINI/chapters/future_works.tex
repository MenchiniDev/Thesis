\section{Future Works}
\label{sec:future-works}

Sebbene i risultati ottenuti rappresentino un avanzamento significativo nella
navigazione autonoma per rover lunari basati su Reinforcement Learning, esistono
numerose possibilità di estendere il lavoro svolto sia dal punto di vista del sistema di
percezione, sia dal punto di vista del controllo e della cooperazione multi–robot.
Di seguito vengono discusse le direzioni principali per sviluppi futuri.

\subsection{Estensione a piattaforme umanoidi o quadrupedi}

L’architettura sviluppata in questa tesi, pur essendo applicata a un rover a quattro
ruote, è concettualmente compatibile con robot umanoidi o quadrupedi, i quali devono 
affrontare sfide ancora più complesse in termini di bilanciamento, percezione attiva 
e locomozione su terreni irregolari.

Negli ultimi anni, vari lavori hanno mostrato come policy basate su RL profondo siano in
grado di controllare gaits altamente dinamici su robot quadrupedi come ANYmal
\cite{lee2020learning, hwangbo2019learning} o su umanoidi di nuova generazione
\cite{xia2022learning}.  
L'adattamento della pipeline sviluppata in questa tesi a tali piattaforme
permetterebbe:

\begin{itemize}
    \item di combinare percezione locale e controllo dinamico, integrando stima della
          pendenza e clustering del terreno con locomozione reattiva;
    \item di testare la robustezza della pipeline percettiva a condizioni cinematicamente
          complesse (rotazioni rapide del torso, oscillazioni dei sensori, ecc.);
    \item di studiare politiche attive di percezione, in cui il robot muove il corpo per
          acquisire osservazioni migliori.
\end{itemize}

L’integrazione con robot come ANYmal sarebbe inoltre particolarmente indicata per
missioni lunari o marziane dove è necessario superare ostacoli impossibili per un rover
tradizionale.

\subsection{Policy collaborative per mappatura e navigazione cooperativa}

Un’estensione naturale del lavoro riguarda l’impiego di più robot che condividono
informazioni tra loro al fine di migliorare sia la mappatura che la navigazione.
Nel contesto del RL multi–agente, diverse architetture collaborative hanno dimostrato
di migliorare l’esplorazione e ridurre l’incertezza del terreno, tra cui:

\begin{itemize}
    \item \textbf{QMIX} \cite{rashid2018qmix}, un metodo di value decomposition per
          coordinamento decentralizzato;
    \item \textbf{MADDPG} \cite{lowe2017multi}, particolarmente adatto ad ambienti
          continui e cooperativi;
    \item \textbf{MAPPO} \cite{yu2022surprising}, oggi lo standard per scenari
          multi–robot complessi con politiche centralizzate durante l’addestramento e
          decentralizzate in esecuzione.
\end{itemize}

L’utilizzo di più rover dotati della pipeline di percezione proposta in questa tesi
permetterebbe di:

\begin{itemize}
    \item ampliare la conoscenza del terreno tramite \textit{shared local mapping};
    \item ridurre l’incertezza di classificazione delle pendenze grazie ad osservazioni
          multiple provenienti da angoli differenti;
    \item costruire mappe di rischio con validazione incrociata tra agenti;
    \item coordinare traiettorie sicure, evitando situazioni in cui più robot entrano
          contemporaneamente in zone ad alta pendenza.
\end{itemize}

Tali approcci collaborativi rappresentano un passo chiave per missioni lunari in cui
più robot operano simultaneamente su vaste aree inesplorate.

\subsection{Riconoscimento di crateri come ostacoli strutturali}

Uno dei limiti della pipeline attuale è l’incapacità di identificare crateri profondi,
che costituiscono una classe di ostacoli significativamente diversa rispetto alle
pendenze positive.  
Il rilevamento dei crateri è noto essere un problema complesso: lavori recenti mostrano
che i metodi tradizionali basati su edge-enhancement e template matching non sono
sempre affidabili \cite{silburt2019lunar}, mentre modelli di deep learning per negative
obstacle detection si sono dimostrati efficaci su robot terrestri
\cite{macenski2020neural}.

Essendo i crateri caratterizzati da \textit{slopes negative}, un’estensione naturale
della pipeline potrebbe comprendere:

\begin{itemize}
    \item una fase di clustering dedicata alle regioni depresse della point cloud;
    \item un modello RANSAC invertito per stimare superfici concave;
    \item l’integrazione di feature geometriche su scala più ampia;
    \item una reward function che penalizzi l’avvicinamento a depressioni ad alto
          gradiente negativo.
\end{itemize}

L’identificazione robusta dei crateri risulterebbe essenziale per garantire la
sicurezza del rover in scenari reali.

\subsection{Aggiunta di una seconda camera per migliorare percezione e robustezza}

Attualmente il rover è equipaggiato con una singola camera RGB-D posta a circa
20\,cm dal suolo, il che limita notevolmente:

\begin{itemize}
    \item la distanza massima di percezione;
    \item la capacità di identificare crateri e variazioni profonde del terreno;
    \item la robustezza alla perdita temporanea di informazione dovuta al rumore.
\end{itemize}

Una possibile estensione consiste nell’aggiunta di una seconda camera montata più in
alto, come suggerito da diverse ricerche sulla percezione multi-view
\cite{zhou2018stereo}.  
I benefici principali includerebbero:

\begin{itemize}
    \item \textbf{sensor fusion} per filtrare il rumore e ottenere misure di profondità
          più stabili;
    \item \textbf{maggiore look-ahead distance}, fondamentale per anticipare ostacoli
          e pianificare traiettorie più sicure;
    \item \textbf{capacità di individuare crateri}, poiché l’angolo di vista
          dall'alto migliorerebbe la percezione di depressioni nel terreno.
\end{itemize}

L'integrazione di due camere richiederebbe una pipeline di calibrazione e fusione
(in stile extrinsic + depth fusion) ma permetterebbe un miglioramento sostanziale
rispetto alla configurazione attuale.

