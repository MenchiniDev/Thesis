\begin{abstract}

Questa tesi affronta il problema della navigazione autonoma di un rover in ambienti lunari 
mediante lo sviluppo di una pipeline integrata che combina modelli 
di Reinforcement Learning con metodi geometrici deterministici a basso costo computazionale. 
L’obiettivo principale è valutare se agenti RL, addestrati in \textit{Isaac Sim} 
tramite PPO, SAC e Recurrent PPO e ottimizzati mediante 
\textit{curriculum learning} possano fornire soluzioni più robuste, 
adattive e generalizzabili rispetto ai planner deterministici 
tradizionali, considerando sia la qualità dei percorsi generati 
sia la capacità di reagire a variazioni improvvise del terreno.

La pipeline percettiva è basata sulla stima della pendenza ottenuta 
da mappe di profondità e su un algoritmo di clustering geometrico 
interpretabile, progettato per sostituire o affiancare approcci classici 
di slope estimation. A tal fine viene proposto un confronto sistematico 
tra metodi deterministici rule--based e tecniche derivate da features 
apprese, valutando stabilità, continuità del segnale e impatto sul comportamento 
dei modelli RL.

Un aspetto centrale del lavoro riguarda i vincoli dell'\textit{edge computing}: 
sia le politiche RL sia la pipeline di percezione sono state sviluppate 
per funzionare su piattaforme a risorse limitate, 
come Raspberry Pi 5 e sistemi embedded analoghi. Questo ha richiesto 
scelte architetturali orientate alla minimizzazione del consumo energetico, 
alla riduzione della latenza, e all’impiego di pipeline numeriche leggere, 
prive di componenti neurali ad alto costo computazionale.

I modelli addestrati in simulazione sono stati valutati tramite metriche di stabilità del training, 
capacità di generalizzazione e robustezza dinamica, quindi trasferiti su piattaforma reale tramite ROS2, 
dove sono state analizzate prestazioni, 
latenza end--to--end e comportamento emergente in scenari fisici.

I risultati mostrano che l’integrazione tra RL, percezione geometrica ottimizzata 
per l’edge computing e simulazione realistica consente al rover di 
prendere decisioni più reattive e flessibili rispetto ai metodi deterministici, 
pur mantenendo un budget di calcolo ed energia compatibile con missioni robotiche in ambienti lunari. 
Il lavoro fornisce infine indicazioni su possibili estensioni future, 
incluse strategie di \textit{sim-to-real} più affidabili e nuove modalità di integrazione tra modelli RL e percezione tradizionale.


\end{abstract}

\chapter{Introduction}


\section{Motivation}

Negli ultimi anni la prospettiva di una presenza umana stabile sulla Luna ha riportato 
al centro dell’attenzione la necessità di sistemi robotici capaci di operare in autonomia 
su terreni accidentati, scarsamente illuminati e caratterizzati da condizioni ambientali 
extreme. Missioni come Artemis e i programmi di esplorazione robotica dell’ESA 
prevedono l’impiego di rover dotati di capacità di navigazione autonoma, 
in grado di muoversi con affidabilità anche in assenza di teleoperazione continua.  

L’autonomia non rappresenta soltanto un fattore di efficienza, ma un requisito operativo. 
I ritardi nel segnale Terra–Luna, la possibile perdita temporanea di comunicazione 
e la natura imprevedibile del terreno rendono impraticabile una guida remota 
di tipo reattivo. I rover devono quindi essere in grado di percepire l’ambiente circostante, 
valutare la traversabilità del terreno, generare traiettorie sicure e reagire in modo dinamico 
a ostacoli e variazioni improvvise.  

Tali capacità devono essere realizzate nel rispetto di vincoli tecnologici molto stringenti. 
I sistemi computazionali installati sui rover planetari sono tipicamente caratterizzati 
da consumi energetici ridotti, risorse limitate e assenza di GPU ad alte prestazioni. 
È dunque necessario sviluppare algoritmi capaci di adattarsi all’ambiente, 
ma sufficientemente leggeri da funzionare in tempo reale su piattaforme embedded.  

In questo scenario, l’integrazione tra metodi di percezione geometrica e 
Reinforcement Learning rappresenta un’opportunità di grande interesse: 
i primi offrono interpretabilità, robustezza e costi computazionali contenuti, 
mentre il RL permette al rover di apprendere strategie di navigazione efficaci, 
sfruttando informazioni parziali e dinamiche complesse. 
Questa tesi si inserisce proprio in questo contesto, con l’obiettivo di 
comprendere se e in che misura un approccio ibrido possa risultare vantaggioso 
rispetto alle pipeline deterministiche impiegate tradizionalmente in robotica planetaria.


\section{Problem Statement}

Il problema affrontato in questa tesi riguarda la navigazione autonoma di un rover 
in scenari lunari parzialmente osservabili, caratterizzati da ostacoli non strutturati, 
pendenze irregolari e condizioni di sensorizzazione limitate.  
La natura stessa dell’ambiente impone alla politica di controllo di prendere 
decisioni basandosi su osservazioni locali e rumorose, provenienti da una singola 
camera RGB–D posta a bassa altezza da terra.  
Il rover non dispone di una mappa globale né di un modello preventivo del terreno, 
e deve quindi reagire a partire da informazioni strettamente locali, 
integrando percezione e controllo in modo continuativo.

Il problema può essere formulato come un processo decisionale 
in condizioni di parziale osservabilità, in cui l’obiettivo del rover è 
raggiungere un target evitando ostacoli e situazioni pericolose 
quali pendenze eccessive, ribaltamenti o collisioni con rocce.  
A questi aspetti si aggiungono vincoli computazionali stringenti: 
sia la pipeline percettiva sia il modello di controllo devono essere 
eseguiti in tempo reale su dispositivi embedded con risorse limitate.  

La domanda centrale è quindi se sia possibile progettare una pipeline percettiva
sufficientemente affidabile, interpretabile ed efficiente da poter alimentare 
una politica RL, e se quest’ultima possa apprendere un comportamento di navigazione 
robusto e generalizzabile a diverse configurazioni del terreno, risultando più 
adattiva rispetto ai planner deterministici classici.  
La tesi mira inoltre a esplorare la capacità dei modelli RL di sfruttare 
forme diverse di memoria (esplicita o implicita) per compensare la 
ridotta osservabilità del contesto.


\section{Contributions}

Il lavoro presentato in questa tesi offre un’indagine sistematica 
sull’integrazione tra metodi deterministici di percezione geometrica 
e tecniche di Reinforcement Learning per la navigazione autonoma in ambienti lunari simulati e reali.  
La pipeline proposta introduce una stima della pendenza basata su clustering geometrico 
e fitting di piani tramite RANSAC, sviluppata con l’obiettivo di fornire una componente 
percettiva affidabile e computazionalmente compatibile con dispositivi embedded.  

Parallelamente, sono stati progettati e addestrati diversi agenti di RL, 
inclusi PPO, SAC e una variante ricorrente di PPO, all’interno di Isaac Lab, 
con l’ausilio di tecniche di curriculum learning che hanno permesso di modellare 
progressivamente la complessità dell’ambiente.  
La valutazione sperimentale ha analizzato in modo approfondito l’impatto della memoria, 
sia nella forma del frame stacking sia attraverso architetture ricorrenti, 
mettendo in evidenza il ruolo cruciale dell’accumulo temporale di informazione 
in contesti caratterizzati da osservabilità limitata.

Gli agenti addestrati sono stati confrontati in una fase successiva, 
con un modello state-of-the-art basato su world models, 
TD–MPC2, che rappresenta uno dei più avanzati approcci 
model–based attualmente disponibili.  

Infine, l’intera pipeline è stata trasferita su una piattaforma robotica reale 
attraverso ROS2, dimostrando la compatibilità computazionale del sistema 
e verificando il comportamento dell’agente in scenari fisici 
simili a quelli simulati, con particolare attenzione a latenza, 
stabilità operativa e robustezza rispetto al rumore del sensore.

Nel complesso, la tesi mostra come una combinazione attenta di percezione geometrica 
e tecniche di Reinforcement Learning possa risultare efficace anche in contesti 
con risorse computazionali limitate, aprendo la strada a futuri sviluppi 
nell’ambito della navigazione autonoma per missioni planetarie.


\section{Thesis Structure}

La tesi è organizzata nel seguente modo: \newline
Il Capitolo~\ref{ch2_background} introduce i concetti fondamentali legati alla navigazione autonoma, 
al Reinforcement Learning, ai modelli dinamici latenti e al paradigma dei world models, 
incluse le idee che stanno alla base dell’approccio TD–MPC2.
\newline
Il Capitolo~\ref{ch:related} offre una panoramica dei lavori affini, 
collocando il contributo di questa tesi nel panorama delle tecniche esistenti.

Il Capitolo~\ref{chap:hardware} presenta una visione completa del sistema sviluppato, 
dalla piattaforma robotica ai moduli software, mentre il Capitolo~\ref{ch5_perception_pipeline} 
descrive in dettaglio la pipeline percettiva proposta, confrontandola con metodi deterministici preesistenti 
e discutendone limiti e potenzialità.  

Il Capitolo~\ref{ch6_simulation_environment} introduce l’ambiente simulato utilizzato per l’addestramento, 
includendo la generazione procedurale del terreno e degli ostacoli, 
l’integrazione dei sensori simulati e il design della funzione di reward.  
Il Capitolo~\ref{ch7_real_environment} presenta invece l’ambiente reale della Luna Analog Facility dell’EAC, 
mettendo in evidenza le differenze rispetto alla simulazione.  

Il Capitolo~\ref{ch8_training_and_models} illustra la metodologia di addestramento, 
le architetture dei modelli e l’integrazione del world model TD–MPC2.  
I risultati sperimentali sono discussi nel Capitolo~\ref{ch9_results}, 
che analizza sia le prestazioni delle pipeline percettive sia il comportamento 
dei modelli RL e del TD–MPC2 in compiti di navigazione complessi.  

Il Capitolo~\ref{ch10_sim2real} descrive il processo di trasferimento su piattaforma reale e le prestazioni del sistema embedded,  
mentre il Capitolo~\ref{ch11_future_works} propone una serie di possibili estensioni del lavoro.  
Infine, il Capitolo~\ref{ch12_conclusions} riassume i principali risultati ottenuti 
e le implicazioni per la navigazione autonoma in contesti extraterrestri.
